\documentclass[a4paper,11pt]{article}
\usepackage[T1]{polski}
\usepackage[utf8]{inputenc}
\usepackage{listings}
\usepackage{xcolor}
\renewcommand{\contentsname}{Table of Contents/Spis treści}
\author{Rafał Gulewski}
\title{\textbf{Mobile Organizer} \\ Functional specification/Specyfikacja funkcjonalna}
\begin{document}
\maketitle
\tableofcontents
\pagebreak
\section{English}
\subsection{General information}
Aplication ,,Mobile-organizer'' is intended for devices with Android system. It allows you to easily save tasks, events and create notes. This help you planning day and keeping your all important notes in one place.
\subsection{Functionality}
The most important functionalities of the application.
\subsubsection{Tasks}
Main function of the aplication is making tasks. User can make task and subtasks. There is information about completion of task.
\paragraph{deadline}
For every task user can set deadline. Task with setted deadline are more important and more visible for user.
\subsubsection{Calendar}
In calendar user can search next deadline in more clearly way. He can also add task or note about event.
\paragraph{Adding tasks} - In calendar user can add task automaticly setting deadline date.
\paragraph{Adding events} - In calendar user can add information of event. Event can be one-time event but it also can be cyclical: weekly, monthly, yearly, or on selected days of the week.
\subsubsection{Notifications}
Aplication sends notifications to user about upcoming events and deadlines.
\subsubsection{Notes}
Aplication allows user to save notes. This function is similar to simple notebook but it allows better sorting of notes. User also can make notes related directly with tasks or events.
\subsubsection{Profile}
User profile is place where user can check his statistics: number of completed tasks, number of overdue task. This place also allows to go to the application options.
\subsubsection{Personalization}
User has the following options of personalization.
\begin{itemize}
\item User can change color of aplication theme, by choosing one of avaible theme in options.
\item User can change aplication icons.
\item User can personalize content of notifications and their frequency. 
\end{itemize}
\section{Polski}
\subsection{Ogólne informacje}
Aplikacja ,,Mobile-organizer'' przeznaczona jest na urządzenia z systemem Android. Pozwala na łatwe zapisywanie zadań, wydarzeń i tworzenie notatek, 
co pozwala na łatwiejsze planowanie dnia i trzymanie zapisek w jednym miejscu.
\subsection{Funkcjonalności}
Poniżej omówiono najważniejsze funkcjonalności aplikacji.
\subsubsection{Zadania}
Główną funkcją aplikacji jest tworzenie zadań. Użytkownik może utworzyć zadanie, jak i podzadania do zadania głównego. Przy każdym zadaniu znajdują się informacje o ukończeniu zadania.
\paragraph{deadline}
Do każdego zadania użytkownik może ustalić deadline. Zadania z ustawionym deadlinem są uznawane za ważniejsze i są lepiej widoczne dla użytkownika.
\subsubsection{Kalendarz}
Kalendarz jest miejscem pozwalającym przejrzeć deadline'y zadań, oraz dodawać notatki o wydarzeniach.
\paragraph{dodawanie zadań} - W kalendarzu użytkownik może dodawać zadania. Automatycznie dobierając datę deadline'u.
\paragraph{dodawanie wydarzeń} - W kalendarzu użytkownik może dodać wydarzenie. Wyderzenie może być jednorazowe jak i też powtarzać się cykliczne co tydzień, co miesiąc, bądź w ustalonych dniach w tygodniu.
\subsubsection{Powiadomienia}
Aplikacja informuje użytkownika o zbliżających się wydarzeniach jak i najbliższych deadlinach zadań.
\subsubsection{Notatki}
Ostatnią z najważniejszych opcji aplikacji są notatki. Użytkownik tak samo jak zadania może dodawać notatki. Notatkę można dodać do danego zadania, bądź wydarzenia, ale również oddzielnie w menu notatek. Notatka posiada tytuł i datę utworzenia, co pozwala lepsze sortowanie na sortowanie. Oprócz tego możliwe jest grupowanie notatek katalogami.
\subsubsection{Profil}
Profil użytkownika jest miejscem gdzie użytkownik może sprawdzić swoje statystyki: liczbę wykonanych zadań, ilość zaległych zadań. Może on również przejść do personalizacji.
\subsubsection{Personalizacja}
Użytkownik posiada wiele opcji personalizacji aplikacji.
\begin{itemize}
\item Użytkownik może zmieniać kolor motywu wybierając jeden z dostępnych w ustawieniach.
\item Użytkownik może dostowywać ikony aplikacji.
\item Użytkownik może dostosowywać treść powiadomień i ich częstotliwość. 
\end{itemize}
\end{document}